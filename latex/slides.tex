% COMO COMPILAR E EXECUTAR ESSE SCRIPT LATEX:
% pdflatex nomeArq.tex
% mupdf nomeArq.pdf

% muda a cor do slide...
\documentclass{beamer}
\usepackage[utf8]{inputenc}

\begin{document}
\begin{frame}{A white frame}
\centering olá como vai tú?
\end{frame}

% Change all subsequent frames to violet
\setbeamercolor{background canvas}{bg=violet!20}
\begin{frame}{A violet frame}
\end{frame}

\begin{frame}{This frame is also violet}
\end{frame}

% But this frame only will be yellow: note { ... } around
% the \setbeamercolor and the frame to limit the scope 

{\setbeamercolor{background canvas}{bg=yellow!20}
\begin{frame}{This frame is yellow}
\end{frame}
}

\begin{frame}{Subsequent frames will be violet}
\end{frame}

% Tenta um black:
\setbeamercolor{background canvas}{bg=black}
\begin{frame}{\color{white}A BLACK frame}
\centering {\color{white}Aqui a letra é branca}
\end{frame}


\end{document}


% IMAGENS EM BACKGOUND:
% \documentclass{beamer}
% \usepackage{tikz}

% \usebackgroundtemplate{%
% \tikz\node[opacity=0.3] {\includegraphics[height=\paperheight,width=\paperwidth]{ctanlion}};}

% \begin{document}
% \begin{frame}
% CTAN lion drawing by Duane Bibby.
% \end{frame}
% \end{document}

% outro exemplo: 

% \documentclass{beamer}
% \usepackage{tikz}

% \usebackgroundtemplate{%
% \tikz[overlay,remember picture] \node[opacity=0.3, at=(current page.center)] {
%    \includegraphics[height=\paperheight,width=\paperwidth]{example-image-a}};
}

% \begin{document}

% \begin{frame}
% Background transparent image, centered on slide
% \end{frame}

% \end{document}
